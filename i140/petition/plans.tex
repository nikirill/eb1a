\newpage
\subsection*{Statement from \dr on how he intends to continue work in the United States}
\label{plans}
\today
\bigskip

I am the beneficiary of this I-140 Immigrant Petition for an Alien Worker, seeking EB-1A classification as an individual of extraordinary
ability. I have a vast experience in the field of \dpcs, and I intend to
continue carrying out research in this area in the United States.

I am already employed as a postdoctoral researcher and I conduct research in
the area of my expertise. 
Specifically, I analyze the leakage of sensitive information from de-identified
medical data and develop privacy-preserving methods for data sharing.
I began my work in the United States at Cornell University, and then continued
to conduct research at Columbia University and the New York Genome Center, all
of which are the top-ranked research institutions in the world (see
\citewb{offers} for the corresponding employment offer letters).
I also have experience in applied science, as I interned at Microsoft
Research---the research arm of one of the largest technology companies in the
world.

After completing my postdoctoral appointment, I plan to apply for professorship
positions in Computer Science at research universities in the United States.
Having become one of the top experts in the field of \dpcs, I believe that I
have a duty to use my expertise to advance the state of the art in privacy
technologies and to share my knowledge with the next generation of
scientists, engineers, and policy-makers in the United States.
I will also continue my presenting research results at international conferences
and serving as a reviewer for conferences and journals in my field.

My research will continue to focus on strengthening privacy protections for
users and improving the cybersecurity posture of computer systems, including
software-update systems and blockchain networks.
I will work on developing mechanisms for private communication and
the private retrieval of information, which, I believe, will become even more in
demand in the future.
I will also continue my work on developing tools for privacy-preserving
analysis and sharing of medical data, which is a field of growing importance in
the upcoming era of personalized medicine.
The National Strategy to Advance Privacy-Preserving Data Sharing and Analytics
highlights the importance of such tools for responsible scientific
research and innovation, especially in healthcare research where data are highly
sensitive and their sharing is limited by the existing regulations.

Getting the permanent residence in the United States will increase my research
opportunities. For example, many research grants at the National Science
Foundation (NSF) or the National Institutes of Health (NIH) are restricted to
U.S. citizens and permanent residents.
With the permanent residency, I will be able not only to apply for such grants
but, thereby, attract and fund the most talented students.

I would like to see more of my research inventions deployed in real-world
systems by U.S.-based companies, as it happened with my work on metadata
protection that was adopted by Apple in iMessage and by Facebook in Messenger.
Such deployment can require my advisory and engagement that I am 
currently not permitted to provide due to my non-immigrant visa status.
As an alternative career path, I consider starting my own company.
I would specifically target privacy-preserving data analytics---the priority
that the National Science and Technology Council highlights in their national
strategy report.
Unfortunately, my non-immigrant status also restricts my ability start a
business and to translate my research into real-world applications that would
benefit the U.S. population.
Obtaining the permanent residency will lift these limitations and allow me to
maximize the positive impact of my research and to contribute more significantly
to the United States' innovation ecosystem.


Yours faithfully,

\vspace{5em}
\drfull

Address: \\
Tel: \\
Email: \\
Personal website: \url{https://nikirill.com}

\pagebreak