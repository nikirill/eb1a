%!TEX root = ./main.tex
\subsection{Evidence of \drs original scientific and scholarly contributions 
	of major significance to the field}
\label{sec:contributions}

The Beneficiary's field of specialization is \dpcs, which involves researching
methods for protecting secrecy of data and communication.
\drs particular expertise is in designing techniques for protecting metadata and
integrity in encrypted files and communication.

Metadata are auxiliary information, such as the data recipients or the
communicating parties, the algorithms used for encryption, and the length of the
encrypted data. 
Protection of metadata is crucial for ensuring the data and users' privacy.
In 2023, the U.S. government published the \textit{\uline{Executive Order on the
Safe, Secure, and Trustworthy Development and Use of Artificial Intelligence}},
which emphasized the importance of developing privacy-enhancing technologies
\textit{``to protect privacy and to combat the broader legal and societal risks
that result from the improper collection and use of people’s data.''}
Consequently, the National Science and Technology Council developed
\textit{\uline{National Strategy to Advance Privacy-Preserving Data Sharing
and Analytics, 2023}} and \textit{\uline{National Privacy Research Strategy,
2025}} (see \citewb{government} for the executive order and the strategy
reports).
\drs work is directly aligned with the goals of the National Strategy,
specifically, in its recommendation to develop tools and techniques for
protecting privacy-sensitive information in transit and at rest.

Data integrity is the assurance that the data has not been tampered with during
transit to a user or when stored on a server.
In his research, \dr developed methods for ensuring data integrity in
software-update systems, which represent the so-called \textit{software
supply-chain}, and in systems for private information retrieval.
The software supply chain is a sequence of steps that a software package goes
through from the initial development to the final deployment on a user's device.
In 2022, in response to the \textit{\uline{Executive Order on America's Supply
Chains}} and the \textit{\uline{Executive Order on Improving the Nation’s
Cybersecurity}}, the Action Plan on Securing Defense-Critical Supply Chains was
developed~\cite{government}.
This directive highlighted the growing risks posed by vulnerabilities in the
software supply chain, such as the insertion of malicious code or unauthorized
modifications to software during its distribution. 
\drs methods address these challenges by introducing robust cryptographic
techniques and blockchain algorithms to verify the authenticity and integrity of
software updates, ensuring that only verified code reaches end users.
These advancements provide means for fortifying critical infrastructure and
safeguarding sensitive systems against evolving cyber threats.

\drs major scientific contributions have been recognized by \textit{experts in
the field worldwide}.
We discuss these contributions in greater detail below.

\subsubsection{Evidence of original scientific contribution: Methods for 
protecting metadata in encrypted data and communications}
\label{sec:purbs}

\dr invented methods for efficient
protection of metadata in encrypted data and communications, which he
presented at the Privacy Enhancing Technologies Symposium and published in the
corresponding proceedings~\cite{purbs}.
When a message is encrypted in storage or sent over an encrypted network
channel, it is commonly supplemented with unprotected auxiliary information,
called \textit{metadata}, which can specify the message's recipient, the
encryption parameters, or the message's length.
It is added to facilitate the message's decryption by the recipient, who,
otherwise, would not know what cryptographic key to use or how to interpret the
ciphertext.
These metadata, however, can be used to infer sensitive information about the
message or the communicating parties, which poses a threat to the users'
privacy.
The techniques invented by \dr facilitate encryption of data that protects
both the content and the metadata, and the recipient still can
efficiently decrypt such a zero-leakage ciphertext.
These methods have been recognized by both the scientific community and
large technological companies as a contribution that solves a
long-standing challenge.

\textbf{\uline{The method for obfuscating the size of encrypted data,
developed by \dr, is already being used by iMessage and Facebook Messenger.}}
Starting with iOS 17.3, iMessage (an instant messaging service developed by
Apple Inc.\@) uses \padme, the padding heuristics developed
by \dr, as a part of the metadata-protection methods, to protect the length of
messages in transit between iMessage users.
Apple specifically chose \padme as it \textit{``strikes an excellent
balance between privacy and efficiency, and preserves the user experience in
limited device connectivity scenarios''} (\textbf{see the announcement by the
Apple Security Engineering and Architecture team, which describes the usage of
\drs invention}~\cite{purbs}).
Furthermore, Facebook Messenger (an instant messaging service developed by Meta
Platforms) uses the same invention of \dr to protect the messages in encrypted
storage, as outlined by the Labyrinth protocol that specifies the encryption of
messages history on the devices of each user's account (\textbf{see the
announcement by Facebook Engineering and the protocol documentation that details
the usage}~\cite{purbs}).
iMessage and Messenger are reported to have over 1.3 and 1 billion
active users each month, respectively (see \citewb{purbs} for evidence of
iMessage and Facebook Messenger's active users' count).

\textbf{\uline{Independent}} expert \jjj, XX at \violetinc, describes the
reasons for adopting \drs work in his letter of support enclosed at
\citewb{letter-jjj} as follows:

\qu{\ldots}

\textbf{\uline{Independent}} expert \ggg, XX at \violetinc, who is a Program
Committee member of the conference where the original publication on \drs work
was presented, elaborates on \drs original contribution in his
letter of support enclosed at \citewb{letter-ggg}:

\qu{\dots}

\bbb, XX at \reduni, a leading expert in privacy-enhancing technologies, describes the
significance of \drs work from the scientific perspective (see \citewb{letter-bbb}):

\qu{\textbf{Dr. Nikitin's work on reducing metadata leakage from encrypted files
and communications is one of his most influential contributions}.
Dr. Nikitin studied a fundamental problem in the existing encryption schemes:
standard formats for encrypted data (aka ciphertexts) expose auxiliary
information (aka metadata), including encryption suites used and payload length.
This exposure can be exploited by traffic analysis, de-anonymization, website
fingerprinting, and many other attacks, and has thus been a focus of much
research in the security and privacy community. 
Instead of the previous, fragile approaches that aimed to distinguish between 
sensitive and not-sensitive metadata, \textbf{Dr. Nikitin proposed a radical
innovation of leaving no unencrypted metadata whatsoever in the ciphertexts. 
While ideal from the privacy perspective, this approach faced serious
efficiency challenges because recipients of encrypted data would not know what
algorithm or cryptographic key to use to decrypt it.
To address this challenge, Dr. Nikitin invented a new way of handling
decryption.}
He proposed decoding techniques that enable a ciphertext recipient
to, first, efficiently find and decrypt the auxiliary markers, then decrypt the
data. 
\textbf{This innovation has made it significantly more difficult for adversaries
to perform traffic analysis or infer sensitive information from encrypted data
at rest, greatly improving protection for sensitive files and communications.}}

\textbf{\uline{Independent}} expert \eee, XX at \purpleinc, and YY at \cyanuni, and a recognized
leader in cryptography and privacy-enhancing technologies, describes the impact
of \drs work from the industrial perspective in his letter of support enclosed
at \citewb{letter-eee}, stating:

\qu{Dr. Nikitin's innovation in iMessage is a technique for obfuscating the
length of encrypted data. Most encryption algorithms nowadays preserve the
length of data when encrypting it. 
For example, the word ``yes'' would commonly be encrypted with three symbols,
whereas the word ``no'' with two symbols. 
This is obviously a privacy issue in the messaging context because, even when
encrypted, the two words would be easily distinguishable. 
Dr. Nikitin proposed a padding technique that struck the optimal balance between
the size protection and the induced bandwidth overhead. 
\textbf{Minimizing the overhead while providing the best possible protection is
critical, as a system like iMessage might be exchanging billions of messages
every day. 
The fact that Dr. Nikitin's innovation provides this balance and that it is
directly translated into deployment by a major technology company underscores
the real-world impact of his work and its importance in securing communications
at scale.}}

Further highlighting the impact of \drs contributions, expert
\textbf{\uline{independent}} letter of support from \fff, XX at \whiteuni, and a renown researcher in security and privacy of
large-scale distributed computer systems states (see \citewb{letter-fff}):

\qu{Kirill is known for his scientific contributions in data privacy and,
specifically, for his work on metadata protection.
\textbf{In his paper titled ``Reducing metadata leakage from encrypted files and
communication with PURBs'', published in the Proceedings on Privacy
Enhancing Technologies, Kirill presented techniques for both obfuscating the
length of encrypted content and protecting encryption metadata. 
This work is important as the length of an encrypted payload can reveal
important information about content; yet protecting it in an efficient way is a
non-trivial task because digital objects can radically differ in size.}
\medskip \\
For example, one user might send a short email of less than 1000 characters in
length, while another user may wish to download a high-definition movie one
million times larger than the email.
It is grossly inefficient, and therefore impractical, to require all
communications to be of the same length.
In this paper Kirill developed padding techniques which ensure messages
conform to a small set of lengths which maximize privacy while ensuring
practicality.
The technique has been adopted by several major tech companies including Apple and Meta.
\medskip \\
One line of my own research work concerns anonymity networks.
Such networks allow users to communicate without revealing the identities of the
communicating parties to the network operator.
As a result, these networks require cryptographic methods and system engineering
designs which do not expose the identities of either the sender or the receiver.
Such systems have traditionally focused on direct, one-to-one communication.
\textbf{Kirill has developed techniques which can encrypt a message which is
readable by multiple recipients without revealing who those recipients are.
This approach is therefore more scalable, opening up new opportunities in future applications.}}


\subsubsection{Evidence of original scientific contribution: Novel techniques for private information retrieval}
\label{sec:apir}

Private information retrieval (PIR) is a paradigm in which a user can
download data from a server without revealing which data are being retrieved.
PIR is one of the highlighted research areas in \uline{The National Strategy to
Advance Privacy-Preserving Data Sharing and Analytics} (see page 18 of the
strategy report enclosed at \citewb{government}).
\drs work is the first to develop a solution to the setting where the
server is fully malicious, \ie, the server can arbitrarily deviate from the
correct behavior, hereby violating the user's privacy.
\textbf{This advancement brings the PIR technology closer to practical
deployment in real-world systems.}
The developed techniques were published in the proceedings of the USENIX
Security Symposium~\cite{apir}, one of the top conferences in the field, and
were immediately recognized by the scientific community.
The manuscript has been cited over 30 times in the first year after its
publication and has spurred a series of follow-up works. 
Here is a quote from one of the follow-up works, published in another major
conference of the field, that builds on top of \drs contributions:

\qu[Dietz, M., Tessaro, S. Fully malicious authenticated PIR. In Annual
International Cryptology Conference 2024.]{Only very recently,
[Colombo, \uline{Nikitin} \etal] initiated the study
of selective-failure attacks in the context of PIR.}


\textbf{\uline{Independent}} expert \ccc, XX at \blueuni, and an expert in
secure computation, blockchain, and privacy-enhancing technologies, explains the
significance of \drs work on PIR (see \citewb{letter-ccc} for the full letter):

\qu{Dr. Nikitin has made significant, original contributions to the
field of data and computer privacy, particularly through his research on Private
Information Retrieval (PIR). 
PIR enables users to retrieve data from a computer database without revealing
what data they are accessing. \textbf{This technology is crucial for improving
the privacy guarantees of online communication, for example, in instant
messaging applications, such as WhatsApp.}
Previously, research in PIR assumed that database servers always followed the
protocol perfectly.
In real-world applications, however, a corrupted or compromised server can
violate the protocol and attempt to break the user's privacy. Dr. Nikitin has
demonstrated that such a server can alter its response to a user's query, ob-
serve whether the user accepts this response, and, subsequently, infer what data
have been accessed. \textbf{Dr. Nikitin's proposed schemes are the first
efficient solution to ensure the privacy of information retrieval even when the
database server is malicious. This achievement bridges the gap between
theoretical PIR protocols and their practical security in real-world
scenarios.}}

\textbf{\uline{Independent}} expert \eee, XX at \purpleinc and YY at \cyanuni, states (see
\citewb{letter-eee}):

\qu{Dr. Nikitin's work on protecting user access patterns follows a long line of
research on private information retrieval (PIR). PIR is a set of techniques for
enabling a computer user to fetch an item from a database without revealing to
the database which item it is. \textbf{While PIR has been extensively studied in
academic settings, all the prior approaches have been unsuitable for real-world
deployment due to being insecure in the adversarial setting. Dr. Nikitin's work
is the first to demonstrate how to make such protocols secure even when the
database operator actively attempts to break the user's privacy.} These security
properties are crucial in applications requiring strong guarantees, such as
secure cloud storage and privacy-preserving search. \textbf{Dr. Nikitin's
contributions bring the PIR technology significantly closer to practical
deployment, enabling robust privacy protections in real-world systems.}}

\subsubsection{Evidence of original scientific contribution: Securing software-update systems}
\label{sec:chainiac}

As highlighted by the \emph{Executive Order on America's Supply Chains}, the
\emph{Executive Order on Improving the Nation’s Cybersecurity}, and the
\emph{Action Plan on Securing Defense-Critical Supply Chains},
software supply chains, and, specifically, software-update systems, are the
critical component of the nation's infrastructure, and they are also a potential
target for cyberattacks.
\dr has developed CHAINIAC, a framework for securing software-update
systems via decentralization and the application of blockchain technology.
The framework was presented and published in the proceedings of the USENIX
Security Symposium~\cite{chainiac}.
The publication has gained significant attention in the scientific community and
has already gathered impressive \textbf{185+ citations} by scientists
across the world~(see the Google Scholar profile of \dr at \citewb{gscholar}).

\textbf{\uline{Independent}} expert \ddd, XX at \yellowuni and a leading expert
in verifiable computation and distributed systems, explains the national
importance of \drs work for the United States (see \citewb{letter-ddd} for
his letter of support):

\qu{\textbf{Dr. Nikitin is one of the world experts in security of software-update systems, a research area of national importance to the United
States. Without robust software update, adversaries can target the software
supply chain. This has been demonstrated by recent attacks. 
For example, government agencies were breached through SolarWinds's software.}
The key point is that, unfortunately, software-update systems are a lucrative
target for malicious actors because compromising a single access point can
enable them to distribute malware to tens of thousands of companies and hundreds
of millions of users. 
\textbf{Failures in the software-update process can also lead to the disruption
of critical services.
For example, the CrowdStrike-related outage several months ago resulted in
grounded flights, halted governmental services, and closed banks with the
estimated worldwide financial damages of at least \$10 billion; it was caused by
faulty software update. Hence, it is of paramount importance to design robust
and secure mechanisms for the software supply chain.}
\medskip \\
\textbf{As a response to this critical challenge, Dr. Nikitin developed an
innovative framework, named CHAINIAC, that was the first to leverage
decentralization and transparency to eliminate single points of failure and to
enforce integrity in the software-release pipeline.}
At a high level, the framework secures each step of the software production
process, from the development of the source code to the installation of the
corresponding update on a user's device. 
Among other techniques, the framework introduced the concepts of collectively
verified builds and skipchains. Prior artifact-verifiability approaches provided
the guarantee that a given source code could be deterministically compiled into
some binary but did not establish any binding between the source code and the
actual release delivered to user devices.
CHAINIAC's innovation was to employ multiple servers that independently compiled
a binary and then attested to a single valid release result that end users can
trust. 
By leveraging skipchains, a novel data structure that Dr. Nikitin designed,
CHAINIAC implemented a public release log that deflected targeted attacks on
high-profile individuals.
\textbf{This work of Dr. Nikitin constituted a major contribution to the field
and influenced the design of multiple follow-up architectures, including
Google's Binary Transparency project.}}

The innovation of \drs work is further explained by \aaa, XX at Black University, in his letter of support enclosed at
\citewb{letter-aaa}:

\qu{\ldots}

\textbf{The final evidence of the significance of \drs work on securing the
software supply chain is that \uline{it has been included in the curriculum of
graduate-level courses at multiple universities in the United States and
abroad}}.
These universities are the University of Chicago, the University of
California, San Diego, the University of Illinois at Urbana-Champaign, and the
Technical University of Munich, Germany (see \citewb{chainiac} for copies of
the curricula).


\subsubsection{Evidence of original scientific contribution: Improving the efficiency of blockchain networks}
\label{sec:blockchain}

Blockchain is a novel technology that enables distributed parties to
perform shared operations without the need for a trusted intermediary.
Blockchain networks rely on this technology to implement crucial services, such
as digital finance, supply chain management, and identity management.
Scaling such networks to support a large number of users and a high volume of
transactions, however, is a challenging task.
The recent \uline{Executive Order on Strengthening American Leadership
in Digital Financial Technology} highlighted the importance of the
responsible growth and use of digital assets and blockchain technology for
supporting innovation and economic development in the United States.

\dr has made two major contributions to the field.
In his work presented at the IEEE Symposium on Security and Privacy, \dr
demonstrated how to drastically increase the throughout of blockchain networks
by changing the approach to transaction verification (see the details in
\citewb{blockchain}).
\dr demonstrated that, instead of requiring all nodes in a network to
verify each transaction, it would be more efficient to allow a single untrusted
prover node to generate a cryptographic proof for the validity of multiple
transactions and let the other nodes in the network verify that proof.
It turned out to be much more efficient than the traditional approach and
became the first ever instance of verifiable-computation techniques being used
to improve the efficiency of a system.
The second contribution of \dr was the design of novel data structure named
skipchain.
Unlike a blockchain that guarantees that only the past blocks are unchanged, a
skipchain facilitates both backward and forward verification.
Furthermore, this verification is significantly more efficient than in a regular
blockchain due to the skipchain's structure.

\textbf{\uline{Independent}} expert \ddd, XX at \yellowuni, explains the
practical importance of \drs original contributions to blockchain research (see
\citewb{letter-ddd}):

\qu{Another influential work by Dr. Nikitin that I am closely familiar with
showed how to improve the performance of Replicated State Machines (RSMs) by
using outsourced verifiable computation. Dr. Nikitin demonstrated that, in a
distributed system where multiple computer nodes executed the same operations,
it could be more efficient for a single node to execute an operation, while
generating a proof of correct execution, and to convince the other nodes of this
correctness by letting them verify the proof. 
\textbf{To achieve the required efficiency, Dr. Nikitin co-developed multiple
complex cryptographic techniques to reduce the cost of proof generation and
verification. This was a groundbreaking result because the research community 
had previously considered outsourced verifiable computation to be a 
high-overhead tool that inevitably caused efficiency decline.}
The demonstration that this tool could, instead, improve efficiency was a 
landmark advancement.%
\medskip\\
\textbf{The result above also had a profound practical impact. The modern
example of RSMs are blockchains systems, a recent technology that has 
applications in finance, governance, and regulation. Dr. Nikitin showed how his
techniques could be applied to Ethereum, the second largest cryptocurrency and a
platform with the market cap of \$400 billion. Concretely, his techniques could
increase the throughput of the Ethereum network fivefold which would 
translate in millions of dollars on saved transaction fees.} 
Several cryptocurrency solutions later adopted and deployed this approach.
Retaining researchers, such as Dr. Nikitin, with a deep expertise in new
developing technologies is critical for the United States to maintain its
position as the world technological leader.
}

\textbf{\uline{Independent}} expert \ccc, XX at \blueuni describes the key contributions of \drs work as follows (see \citewb{letter-ccc}):

\qu{Dr. Nikitin has also advanced blockchain research by demonstrating how to
integrate state-of-the-art verifiable outsourced computation into permissioned
or permissionless blockchains.
In essence, a blockchain achieves strong integrity protection through the
redundant verification of all blocks and the transactions they contain by
numerous---e.g., thousands or tens of thousands---of independent decentralized
participants. 
\textbf{The key idea of Piperine, the system that Dr. Nikitin and his
collaborators from Microsoft Research designed, is to reduce this large
redundant verification cost by allowing a single untrusted prover to produce a
verifiable cryptographic proof of the correctness of a block of transactions and
the many independent verifiers to merely verify these proofs, instead of the
transactions themselves.}
While simple in concept, the key technical challenges are in bridging
the huge remaining efficiency gap between state-of-the-art outsourced
computation and direct re-execution, as well as the many limitations and
impedance mismatches between current outsourced computation methods and the
requirements of the blockchain context. 
\textbf{I believe that Piperine represents a major contribution both to
blockchain and outsourced computation research, and it illustrates Dr. Nikitin's
breadth and independent collaboration abilities in security/privacy research.}}

Finally, \textbf{\uline{independent}} expert \fff, XX at \whiteuni
explains how these contributions are representative of \drs expertise (see
\citewb{letter-fff}):

\qu{Dr. Nikitin has also made notable contributions in the field of blockchains
--- immutable distributed datastores of information.
In his paper titled ``Replicated state machines without replicated execution'',
he demonstrates that the throughput of blockchain networks can be increased by
using verifiable computation.
Specifically, instead of replicating multiple transactions across many nodes, a
blockchain node instead proves the correctness of their cumulative execution and
sends this proof. 
Using this approach, other nodes can verify the proof instead of redoing the
computation, and it turns out this is faster.
\textbf{This is a surprising result and demonstrates Kirill's expertise and
creative thinking in this domain.}
\medskip \\
Another example can be found in his paper titled ``Chainiac: Proactive
software-update transparency via collectively signed skipchains and verified
builds''. 
In this paper Kirill introduces the novel skipchain data structure. Unlike
traditional blockchains, a skipchain facilitates verifiable forward traversal of
blocks and more efficient backward traversal.
\textbf{It is another example of an innovative creation which provides
significant performance improvements, demonstrating that Kirill is a leading
researcher in the field.}}

As clearly outlined in the detailed discussion above, \dr has not only made
original scientific contributions to the field of \dpcs, specializing in
metadata protection, the security of software-update systems, and the efficiency
of blockchain networks, but they have also been contributions of major
significance, including their use by \textit{billions} of users, which exceeds
the requirement for this criterion.
