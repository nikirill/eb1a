%!TEX root = ./main.tex
\subsection{Evidence that \dr has been a judge of the work of others 
	in Data Privacy and Computer Security and adjacent areas}
\label{sec:judging}

\dr has been a reviewer for multiple years for the top conferences and journals 
in the field of \dpcs.
\dr has also reviewed multiple manuscripts in the area of Blockchain research 
thanks to his expertise in the topic and due to the topic's
significant overlap with the science of Data Privacy and Computer Security.
He has served on the Program Committee and provided reviews for, 
among others, the ACM Conference for Computer 
and Communication Security, the USENIX Security Symposium, and 
the IEEE International Conference on Blockchain and Cryptocurrency, which are 
the most prestigious and selective conferences in the field.

\textbf{\dr has judged a total of \numreviews submissions to top international
scientific conferences and journals (see \citewb{reviews} for the collection of
reviews written by \dr)}:
\begin{itemize}
	\item (32 reviews) ACM Conference on Computer and Communications 
	Security (CCS): with an h5-index of 92, it is ranked \#1 by Research.com 
	among Computer Security and Cryptography conferences (see \citewb{venues}
	for the conference ranking by Research.com). \dr served on the Program
	Committee of the conference during the years 2024, 2023, and 2021.
	\item (16 reviews) USENIX Security Symposium: with an h5-index of 92, it is 
	ranked \#3 by Research.com among Computer Security and Cryptography 
	conferences~\cite{venues}. \dr is currently serving on the conference's
	Program Committee during the year of 2025.
	\item (1 review) Annual International Conference on the Theory 
	and Applications of Cryptographic Techniques (Eurocrypt): with an h5-index of 63, it is ranked \#9 by Research.com among Computer Security and 
	Cryptography conferences~\cite{venues}. \dr served as an invited expert-reviewer in 2022.
	\item (6 reviews) IEEE International Conference on Blockchain and 
	Cryptocurrency (ICBC): with an h5-index of 39, it is considered a flagship 
	conference in the emerging field of Blockchain, and specifically Blockchain 
	Security and Privacy~\cite{venues}. \dr served on the conference's Program
	Committee during the year of 2019.
	\item (1 review) IEEE Transactions on Industrial Informatics (TII): with an 
	impact factor of 12.3 and an h5-index of 167, it is ranked \#1 by 
	Research.com among Databases \& Information Systems journals~\cite{venues}.
	\dr was an invited expert-reviewer for a submission on securing blockchain 
	smart contracts.
	\item (2 reviews) IEEE Transactions on Parallel and Distributed Systems 
	(TPDS): with an impact factor of 5.3 and an h5-index of 74, it is a premier 
	venue for articles on distributed systems, software and
	algorithms~\cite{venues}. \dr was invited as an expert in securing
	software-update computer systems.
	\item (2 reviews) International Conference on Research in Computational 
	Molecular Biology (RECOMB): it is ranked \#4 by Research.com among 
	conferences in Genetics and Molecular Biology~\cite{venues}. \dr was invited
	as an external expert to review a submission on privacy of genetic data.
	\item (1 review) International conference on Intelligent Systems for
	Molecular Biology (ISMB): it is the flagship meeting of the International
	Society for Computational Biology. \dr was invited as an external expert to review a submission on quantifying privacy risks of genomic data sharing.
	\item (5 reviews) Workshop on Cryptocurrencies and Blockchains for 
	Distributed Systems (CryBlock): \dr served on the Program Committee of the 
	workshop during the years of 2019 and 2020.
	\item (2 reviews) International Conference on Blockchain and Trustworthy 
	Systems (BlockSys):\\ \dr served on the Program Committee of the 
	conference during the year of 2019.
\end{itemize}

\dr has extensively reviewed for conferences in Computer Security, Data Privacy,
Blockchain, and related fields.
In these rapidly evolving disciplines, most researchers prefer to publish in
conferences over journals.
According to Google Scholar, 15 out of 20 scientific venues with the most cited 
publications in the field of Computer Security and Cryptography are conferences~\cite{venues}.
Most of these conference papers are not merely abstracts but are 
full scientific articles, typically 11-13 pages long, often accompanied by 
substantial supplementary material, and they present significant theoretical 
and experimental advancements in the subject matter.
\textbf{The invitations to serve on the Program Committee of the
top conferences in the field is a recognition of \drs expertise and standing in
the scientific community.}

\textbf{\uline{Independent}} expert \ccc, YY \blueuni, served with \dr on the Program Committee of the ACM Conference on
Computer and Communications Security, a premier conference in the field, which he highlights in his letter of support enclosed at
\citewb{letter-ccc}:

\qu{Another evidence of Dr. Nikitin's recognition as a leading expert in the
field is his role in judging the work of other researchers.
He has already served on the Program Committee for such prestigious conferences
in computer security and privacy as the ACM Conference Computer and
Communications Security, the USENIX Security Symposium, and the IEEE
International Conference on Blockchain and Cryptocurrency.
\textbf{Invitations to join the Program Committee of such conferences are
extended only to internationally renowned scientists, as the members do not only
provide their experts reviews but also participate in the discussion and make
collective final decisions of either accepting or rejecting submissions.
I can confirm that Dr.~Nikitin possesses the highest level of expertise required
for this role, as both he and I were active Program Committee members of the ACM
Conference on Computer and Communications Security.}
His contributions underscore his deep understanding of complex technical
concepts and his ability to evaluate their significance and potential impact. 
Finally, his active participation demonstrates not only his professional
standing among peers but also his dedication to advancing the field of computer
security and privacy.}

\sparagraph{Membership in the Association for Computing Machinery (ACM).}
In recognition of his contribution to the review process of the ACM Conference
on Computer and Communications Security, \dr was invited to become a
professional member of the Association for Computing Machinery (ACM) which he
has accepted (see \citewb{acm_membership} for \drs membership certificate).
\dr is also a professional member of the Institute of Electrical and Electronics
Engineers (IEEE) society, which is the world's largest association of computing
professionals and a leading organization in the field of computer science and
engineering~\cite{acm_membership}.

\sparagraph{An organizer of RECOMB-PRIEQ 2024.}
\dr was one of the organizers of The Satellite Conference on Biomedical Data
Privacy and Equity (RECOMB-PRIEQ) 2024, that targeted challenges in privacy,
security, bias, and fairness in biomedical research.
This was a highly specialized conference for scientists working on privacy of
biomedical data, which \dr was entrusted to co-organize due to his expertise in
data privacy~\cite{recomb-prieq}.

